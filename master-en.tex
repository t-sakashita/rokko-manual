\documentclass[12pt]{article}

\usepackage{graphicx}

\usepackage{amssymb, amsmath}
%\usepackage{amsthm}

\usepackage{mathtools}

\usepackage{xparse}
%\usepackage{xcolor}

\usepackage{subcaption}
\usepackage[skip=0pt]{subcaption}
\captionsetup{compatibility=false}
\captionsetup{subrefformat=parens}
%\captionsetup{skip=\dimexpr\abovecaptionskip-8pt}

\usepackage{tikz}
\usetikzlibrary{matrix,backgrounds}
\usepackage{pgfmath}

\usepackage{draw_dense_matrix}
\usepackage{draw_sparse_matrix}
%\usepackage{draw_dir_structure}

\NewDocumentCommand{\fillhighlightelement}{O{blue!20} m}{%
\draw[draw,fill=#1] (#2.north west)rectangle (#2.south east);}

\usepackage{pbox, cellspace}
\cellspacetoplimit = 6pt
\cellspacebottomlimit = 6pt
%\renewcommand{\arraystretch}{2}%  1 is the default, change whatever you need

% package introduced by authors
\usepackage{url}
\usepackage{listings}
%\usepackage{cleveref}
%\usepackage[strings]{underscore}

%\usepackage[english]{babel}
%\usepackage{underscore}

\newcommand{\smallscr}[1]{\scalebox{0.4}{$#1$}}
\newcommand{\middlescr}[1]{\scalebox{0.55}{$#1$}}
\newcommand\figref[1]{Fig.~\ref{#1}}
\newcommand\figsref[1]{Figs.~\ref{#1}}
\newcommand\subfigref[1]{Fig.~\subref{#1}}
\newcommand\tabref[1]{Table~\ref{#1}}
\newcommand\secref[1]{Section~\ref{#1}}
\newcommand\itemref[1]{(\ref{#1})}

% define thick horizontal line
\newlength\savedwidth
\newcommand\whline{%
    \noalign{\xdef\origarrayrulewidth{\the\arrayrulewidth}%
    \global\arrayrulewidth 3\arrayrulewidth}%
    \hline%
    \noalign{\global\arrayrulewidth\origarrayrulewidth}%
}

\lstdefinestyle{shstyle}{
    language=bash,
    basicstyle=\ttfamily,%\tiny,
    frame=single,
    breaklines=true,                % sets automatic line breaking
    breakautoindent=false,
    breakatwhitespace=false,        % sets if automatic breaks should only happen at whitespace
    deletekeywords={local, test},  % exclude from reserved keyword lists which is hightlighted
    linewidth=0.9\linewidth
}

\DeclareRobustCommand*\template[1]{%
$\left< \text{#1}\right> $}

\lstset{
language=c++,
basicstyle=\small\sffamily,
%numbers=left,
%numberstyle=\tiny,
frame=single,
columns=fullflexible,
showstringspaces=false,
linewidth=0.9\linewidth
}


\begin{document}

\section{Review of matrix representations}\label{sec:matrix_rep}

In this section, we review general concepts of matrix representation for dense and sparse eigensolvers.

\subsection{Representations of dense matrix}

\subsubsection{Serial dense matrix and vector}
In general, we need matrix to store matrix to be eigen-decomposed and eigenvectors and vector to store calculated eigenvalues.

In computer memory, both matrix and vector are realized by one dimensional array.
%For C, it is just one dimensional array.
There are two ways to assign matrix elements to one dimensional array as illustrated in \figref{fig:localizedmatrix}.
In the \figref{fig:localizedmatrix}, the integer values in matrix element means array addresses and blue-colored arrows mean storage order.
In \figref{fig:localizedmatrix}\subref{subfig:localizedmatrix-row-major}, row-major is the indexing way that the elements of each row have contiguous addresses.
In \figref{fig:localizedmatrix}\subref{subfig:localizedmatrix-col-major}, another major, column-major is defined similarly.

%We can do padding leading dimension.
%If you dealing with top left sub-matrix, leading dimension is useful.

\noindent\begin{minipage}{\linewidth}
  \setcaptiontype{figure}
  \centering
  \subcaptionbox{row-major\label{subfig:localizedmatrix-row-major}}
%  [0.4\linewidth]{\localizedmatrixrowmajor{4}{4}}
  [\widthof{$\localizedmatrixrowmajor{4}{4}$}]{$\localizedmatrixrowmajor{4}{4}$}
  \subcaptionbox{column-major\label{subfig:localizedmatrix-col-major}}
%  [0.4\linewidth]{\localizedmatrixcolmajor{4}{4}}
  [\widthof{$\localizedmatrixcolmajor{4}{4}$}]{$\localizedmatrixcolmajor{4}{4}$}
\caption{Example of \texttt{localized\_matrix}}\label{fig:localizedmatrix}
\end{minipage}

\subsubsection{Parallel dense matrix and vector: Block-cyclic representation}
\label{subsubsec:rep-block-cyclic}
%All eigensolvers for dense matrices employ 2D block-cyclic distributed matrices.
To distribute matrix in memory at MPI processes, we need to assign MPI processes in two dimension firstly.
For default, Rokko creates 2D process gird as far nearly square as possible.
Similarly to \texttt{localized\_matrix}, there are two major types for 2D process grid, row-major and column-major.
All dense solvers support both grid majors.
Let us show both major types of $2\times 2$ size 2D process grid by $4$ processes (no.\ $0,1,2,3$) in \figref{fig:2d-grid}.
In this figure, MPI process number is distinguished by its color.
It is noted that MPI process number and $x, y$-coordinates begins at $0$ for both Fortran and C++/C.

\noindent\begin{minipage}{\linewidth}
  \setcaptiontype{figure}
  \centering
  \subcaptionbox{row-major\label{subfig:grid-row-major}}
  [\widthof{\drawgridrowmajor}]{\drawgridrowmajor}
  \subcaptionbox{col-major\label{subfig:grid-col-major}}
  [\widthof{\drawgridcolmajor}]{\drawgridcolmajor}
\caption{Example of 2D grid for 4 processes}\label{fig:2d-grid}
\end{minipage}

%Next, by using 2D process grid, we define block-cyclic matrices.
%According to given a 2D process grid, \emph{block-cyclic matrix} distributed to MPI processes is defined.
\emph{Block-cyclic matrix} is a storage style distributed to given a 2D process grid as squared tiles. 
Paving the whole matrix with the patterns of 2D process grids of given block size, you obtain block cyclic distribution.
\figref{fig:block-cyclic-matrix}\subref{subfig:global-view} is an example of expressing the matrix in the \figref{fig:localizedmatrix}\subref{subfig:localizedmatrix-col-major} as a block-cyclic matrix,
using 2D process grid shown in \figref{fig:2d-grid}\subref{subfig:grid-col-major}.

Each tile is assigned to one of processes in 2D process grid.
The colors of tiles indicate assigned MPI processes.
The size of squared tile is called \emph{block size}.
In \figref{fig:block-cyclic-matrix}, $2\times2$ block size is used.
EigenExa and Elemental support only $1\times1$ block size for computational efficiency.
ScaLAPACK and ELPA support arbitrary block sizes.


\begin{minipage}{\linewidth}
\setcaptiontype{figure}
\begin{minipage}[c]{.5\linewidth}
\centering\drawblockcyclicglobal
\subcaption{global view}\label{subfig:global-view}
\end{minipage}%
\begin{minipage}[c]{.5\linewidth}
\centering\drawblockcycliclocal
\subcaption{local view}\label{subfig:local-view}
\end{minipage}
\caption{Block-cyclic matrix}\label{fig:block-cyclic-matrix}
\end{minipage}


\figref{fig:block-cyclic-matrix}\subref{subfig:global-view} is virtual global view of the block-cyclic matrix.
By pulling out the same color tiles in \figref{fig:block-cyclic-matrix}\subref{subfig:global-view},
we obtain an actual local storage for each MPI process shown in \figref{fig:block-cyclic-matrix}\subref{subfig:local-view}.
For example, the part of process $0$ is obtained by pulling out the red-colored tiles.
Similarly, the part of process $1$ is obtained by pulling out the yellow-colored tiles.

It should be noted that 2D process grid and block-cyclic matrices are used not only for eigensolvers, but also dense linear algebra libraries such as linear equation solver, LU and QR decomposition in general.

\subsection{Representations of sparse matrix}

To give data of matrix to be eigen-decomposed to sparse solvers, there are two representations of sparse matrix.
The first one is generating \emph{CRS (Compressed Row Storage)} and the second one is \emph{matrix-free method}.
We describe these representations in the subsections below.

\subsubsection{CRS (Compressed Row Storage)}

CRS is a matrix representation to store only nonzero entries and their column indices for each row.
CRS is easy, but need extra memory and time for storing.

% The matrix for the matrix in Equation \eqref{eq:sparse-matrix} is represented as CRS representation below:
% \begin{itemize}
% \item \verb|nonzero_rows = [2, 1, 1, 2, 1, 2, 1, 1]|
% \item \verb|nonzero_cols = [0, 1, 3, 3, 1, 2, 0, 2, 4, 4, 3]|
% \item \verb|nonzero_entries = [7.1, 5.2, 6.4, 4.3, 0.4, 0.5, 0.1, 0.5, 0.2, 1.4, 2,3]|
% \end{itemize}
In MPI parallelization for CRS, chunk of rows are assigned for each process.
\figref{fig:parallel-crs} shows example of parallelization for the sparse matrix in Equation \eqref{eq:sparse-matrix}.
\begin{figure}[!htbp]
\DrawParallelCRS{2}{4}
\centering\caption{Parallelization of CRS matrix}\label{fig:parallel-crs}
\end{figure}


\subsubsection{Matrix-free method}
\emph{Matrix-free method} is to implement matrix-product routine and give it to solvers.
This method does not need extra memory to construct a matrix explicitly to be eigen-decomposed, which has possibility for efficient calculation.
%A user need to write matrix-vector product routine.
If simple formula to generate sparse matrix is known, matrix-free method may have an advantage in performance than CRS method.
%However, parallelizing your own matrix-vector product routine may be hard task.
%Each process has matrix elements shown by dark colors surrounded by dashed lines in \figref{fig:parallel-matrix-vector-product}.

\subsubsection{Parallelization for product of sparse matrix and dense vector}
For both CRS and matrix-free method, matrix-vector product is executed at each iteration in sparse eigensolvers.

In MPI parallelization, each process has a chunk of rows of CRS, a chunk both of input vector $\texttt{x}$ and output vector $\texttt{y}$ as shown in \figref{fig:parallel-matrix-vector-product}.
In order to calculate each element of output vector $\texttt{y}$ of matrix-vector product, the chunks of \texttt{x} possessed by the other processes are required.
To send and receive these chunks, MPI communication is used.

For CRS, this communication is automatically performed by eigensolver libraries and thereby a user does not have to care about in parallelization of this product.
%It has to send and receive the other matrix elements.
For matrix-free method, he needs to write this communication explicitly.

\begin{figure}[!htbp]
\DrawSparseMatrixVectorProduct{2}{4}
\centering\caption{Parallelization of matrix-vector product}\label{fig:parallel-matrix-vector-product}
\end{figure}

\end{document}
